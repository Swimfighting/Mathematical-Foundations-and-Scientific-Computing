\documentclass{article}
\usepackage{graphicx} 
\usepackage{amsmath}
\usepackage{subcaption}
\usepackage{float}
\usepackage{changepage}


\title{Group Project Report: Analytical and Numerical Approaches to\\ Ordinary and Partial Differential Equations}
\author{Group 15\\ \quad Ana Williams \quad Man Wong \quad Sameera Jeetun \\ Yun XIE \quad Yu-Ying SU \quad Zekai LIN}
\date{November 2025}
\usepackage[a4paper,top=0.8in,bottom=0.8in,left=0.8in,right=0.8in]{geometry}


\makeatletter
\renewcommand{\maketitle}{
  \begin{center}
    \vspace*{-1cm}
    {\large \bfseries \@title \par}
    \vspace{0.6cm}
    {\large \@author \par}
    \vspace{0.5cm}
    {\large \@date \par}
    \vspace{0.2cm}
  \end{center}
}
\makeatletter


\begin{document}

\maketitle

\renewcommand{\abstractname}{\centering  \large Abstract}
\begin{abstract}
\begin{adjustwidth}{0.5em}{0.5em}
\normalsize
\setlength{\baselineskip}{1.1\baselineskip}
This project applies Frobenius and Fourier series, along with numerical simulations, to analyse an ODE system and a heat-equation model. The series solution for the ODE agrees closely with numerical results near the singular point, while Fourier expansions capture the temperature evolution in the PDE. Together, these methods show how analytical and numerical tools complement each other in understanding differential-equation behaviour.
\end{adjustwidth}
\end{abstract}

\tableofcontents

\section{Introduction}
\hspace*{1.5em}This project investigates two differential-equation problems that illustrate how analytical and numerical methods can be combined to understand the behaviour of both ordinary and partial differential equations. 

In Question A, we analyse a coupled ODE system that contains a singular point at $x = 0$. Using the Frobenius method, we determine the possible exponent pairs, identify the analytic branch compatible with the initial conditions, and derive a four-term power-series approximation. A numerical solution is then used to validate the accuracy of the truncated series near the singular point.

In Question B, we examine heat conduction in a square plate governed by the two-dimensional heat equation with mixed boundary conditions. Through separation of variables, we construct both the steady-state and transient components as Fourier series, and numerical visualisation demonstrates the diffusion of heat toward equilibrium.

\section{Question A: Series solutions for systems of ODEs}

For series solutions for systems of ODEs, we examine the coupled ODE system
\begin{equation*}
\begin{cases}
x\,\dfrac{dy}{dx} - y = z, \\[4pt]
2x\,\dfrac{dz}{dx} - 3z = -xy,
\end{cases}
\end{equation*}
near the expansion point $x_0 = 0$ with the values of initial conditions: $y(0) = a_0$ and $z(0) = b_0$.

\subsection{Part(a): Why $x_0 = 0$ expected to be a singular point}

The system can be rewritten in the form
\[
\dfrac{dy}{dx} = \dfrac{y}{x} + \dfrac{z}{x}, \qquad
\dfrac{dz}{dx} = \dfrac{3z}{2x} - \dfrac{xy}{2x}.
\]
Both right–hand sides contain coefficient fuctions such as
\[
\dfrac{1}{x}, \qquad \dfrac{3}{2x},
\]
which become unbounded as $x \to 0$.
Therefore the point $x = 0$ is a singular point of the system (the coefficients are not analytic at $x=0$).

\subsection{Part(b): Determine the Frobenius exponents $p, q$}

This analysis determines the value of the exponents p and q for the two Frobenius series solutions and compares the lowest powers of x. Accordingly, the Frobenius series are assumed and expressed in the form
\[
y(x)=\sum_{n = 0}^{\infty} a_n x^{n+p}, 
\qquad 
z(x)=\sum_{n = 0}^{\infty} b_n x^{n+q},
\]
where $a_0,\, b_0\neq 0$. Substituting these series into the given system and comparing the lowest powers of $x$ determines the exponents $p,q$.

\subsubsection{Substituting into $x\,\dfrac{dy}{dx} - y = z$}
From
\[
\dfrac{dy}{dx} = \sum_{n=0}^\infty a_n (n+p)x^{n+p-1},
\]
we obtain
\[
x\, \dfrac{dy}{dx} - y
= \sum_{n=0}^\infty a_n (n+p-1)x^{n+p}.
\]
Equating with \(z = \sum_{n=0}^\infty b_n x^{n+q}\), we got that
\[
 \sum_{n = 0}^\infty a_n (n+p-1)x^{n+p}
=\sum_{n = 0}^\infty b_n x^{n+q},
\]
the lowest-order balance is (when $n = 0$)
\[
(p-1)a_0 x^{p} \sim b_0 x^{q}.
\]
Thus either
\[
\text{(A)}\;\; p = q,
\qquad\text{or}\qquad
\text{(B)}\;\; p = 1,\; q > 1.
\]

\subsubsection{Substituting into $2x\,\dfrac{dz}{dx} - 3z = -xy$}
Similarly, from
\[
2x \dfrac{dz}{dx} - 3z
= \sum_{n=0}^\infty (2n+2q-3)b_n x^{n+q},
\qquad
-xy = -\sum_{n=0}^\infty a_n x^{n+p+1},
\]
the lowest-order balance is(when $n = 0$)
\[
(2q-3)b_0 x^{q} \sim -a_0 x^{p+1}.
\]
Hence either
\[
\text{(C)}\;\; q=p+1,
\qquad\text{or}\qquad
\text{(D)}\;\; q=\dfrac32.
\]

\subsubsection{Combining the cases.}
Matching (A),(B) with (C),(D):
\begin{itemize}
    \item (A)+(C) gives $p=q=p+1$, impossible.
    \item (A)+(D) gives 
    \[
    p=q=\dfrac{3}{2}.
    \]
    \item (B)+(C) gives 
    \[
    p=1,\qquad q=2.
    \]
    \item (B)+(D) gives 
    \[
    p=1,\qquad q=\dfrac{3}{2}.
    \]
Recall the lowest-order balance of $x\,\dfrac{dy}{dx} - y = z$ is
\[
(p-1)a_0 x^{p} \sim b_0 x^{q}.
\]
For $p=1$ we have $p-1=0$,  so the coefficient of the term $x^{p}=x$
on the left-hand side vanishes, and the lowest-order become $x^{p+1}=x^2$ or higher. However the lowest-order of right-hand side is $x^{q} = x^{\frac{3}{2}}$ .
Therefore, the choice $p = 1, \, q = \frac{3}{2}$ cannot satisfy the lowest–order
balance in the first equation, and the combination (B)+(D) is inconsistent.
\end{itemize}

\subsubsection{Conclusion}
The system therefore admits two Frobenius-type solutions with exponents
\[
(p,q)=(1,2)
\qquad\text{and}\qquad
(p,q)=\left(\dfrac{3}{2},\dfrac{3}{2}\right).
\]


\subsection{Part(c): Four-term approximate solution}

From part (b) we have found two possible pairs of Frobenius exponents:
\[
(p,q) = (1,2)
\quad\text{and}\quad
(p,q) = \left(\dfrac{3}{2},\dfrac{3}{2}\right).
\]
Since we wish to obtain the general solution near $x = 0$ that only depends on  the initial conditions $y(0) = a_0$ and $z(0) = b_0$, we must examine the
regularity of these Frobenius solutions and check which of them are analytic
at $x_0=0$.

\subsubsection{Analytic versus non-analytic behaviour at $x_0=0$}

The pair $(p,q)=(1,2)$ leads to series of the form
\[
y(x) = a_0 x + a_1 x^2 + a_2 x^3 + \cdots, 
\qquad
z(x) = b_0 x^2 + b_1 x^3 + b_2 x^4 + \cdots,
\]
which involve only integer powers of $x$. Therefore the solution is analytic (at least in a neighbourhood of $x_0=0$).
On the other hand, the pair $(p,q)=(\tfrac{3}{2},\tfrac{3}{2})$ produces series of the form
\[
y(x) = a_0 x^{\tfrac{3}{2}} + a_1 x^{\tfrac{5}{2}} + \cdots, 
\qquad
z(x) = b_0 x^{\tfrac{3}{2}} + b_1 x^{\tfrac{5}{2}} + \cdots.
\]
These expansions contain half-integer powers of $x$, so the corresponding solutions are not analytic at $x_0=0$. Thus, when constructing a regular four-term approximation determined by $y(0) = a_0$ and $z(0) = b_0$, we must choose the Frobenius solution that is analytic at $x_0 = 0$, which is the $(p,q) = (1,2)$ branch.

\subsubsection{Series ansatz for the analytic solution}
We therefore write
\[
y(x) = a_0 x + a_1 x^2 + a_2 x^3 + a_3 x^4 + \cdots,
\qquad
z(x) = b_0 x^2 + b_1 x^3 + b_2 x^4 + b_3 x^5 + \cdots,
\]
where $a_0$ and $b_0$ are free constants and the higher coefficients
$a_1,a_2,a_3,\dots$, $b_1,b_2,b_3,\dots$ are determined recursively
by the differential equations.

\subsubsection{Step 1: Using $x\,\dfrac{dy}{dx} - y = z$}

From
\[
y'(x) = a_0 + 2a_1 x + 3a_2 x^2 + 4a_3 x^3 + \cdots,
\]
we obtain
\[
xy'(x) = a_0 x + 2a_1 x^2 + 3a_2 x^3 + 4a_3 x^4 + \cdots,
\]
and hence
\[
xy' - y 
= (a_0 x + 2a_1 x^2 + 3a_2 x^3 + 4a_3 x^4 + \cdots)
  - (a_0 x + a_1 x^2 + a_2 x^3 + a_3 x^4 + \cdots),
\]
after grouping the terms by powers of $x$, we find that
\[
xy' - y = a_1 x^2 + 2a_2 x^3 + 3a_3 x^4 + \cdots.
\]
Equating this with
\[
z(x) = b_0 x^2 + b_1 x^3 + b_2 x^4 + \cdots,
\]
we obtain, by matching coefficients of each power of $x$,
\[
a_1 = b_0, 
\qquad
2a_2 = b_1,
\qquad
3a_3 = b_2.
\tag{R1}
\]

\subsubsection*{Step 2: Using $2x\,\dfrac{dz}{dx} - 3z = -xy$}

We compute
\[
z'(x) = 2b_0 x + 3b_1 x^2 + 4b_2 x^3 + 5b_3 x^4 + \cdots,
\]
so that
\[
2xz'(x) = 4b_0 x^2 + 6b_1 x^3 + 8b_2 x^4 + \cdots,
\]
and
\[
3z(x) = 3b_0 x^2 + 3b_1 x^3 + 3b_2 x^4 + \cdots.
\]
Thus
\[
2xz' - 3z
= (4b_0-3b_0)x^2 + (6b_1-3b_1)x^3 + (8b_2-3b_2)x^4 + \cdots
= b_0 x^2 + 3b_1 x^3 + 5b_2 x^4 + \cdots.
\]
On the right-hand side we have
\[
-xy = -x(a_0 x + a_1 x^2 + a_2 x^3 + \cdots)
= -a_0 x^2 - a_1 x^3 - a_2 x^4 + \cdots.
\]
Equating coefficients of each power of $x$ yields
\[
b_0 = -a_0,
\qquad
3b_1 = -a_1,
\qquad
5b_2 = -a_2.
\tag{R2}
\]
we obain
\[
b_0 = -a_0.
\]

\subsubsection{Step 3: Solving the recurrence relations}

Substituting this into $(\text{R1})$ gives
\[
a_1 = b_0 = -a_0.
\]
Next, from $(\text{R2})$ we have
\[
3b_1 = -a_1 = a_0
\quad\Rightarrow\quad
b_1 = \frac{a_0}{3},
\]
and then from $(\text{R1})$,
\[
2a_2 = b_1 = \frac{a_0}{3}
\quad\Rightarrow\quad
a_2 = \frac{a_0}{6}.
\]
Similarly,
\[
5b_2 = -a_2 = -\frac{a_0}{6}
\quad\Rightarrow\quad
b_2 = -\frac{a_0}{30},
\]
and hence
\[
3a_3 = b_2 = -\frac{a_0}{30}
\quad\Rightarrow\quad
a_3 = -\frac{a_0}{90}.
\]

\subsubsection{Four-term expansions}

Collecting the coefficients, we obtain the four-term approximations
\begin{align*}
y(x) &= a_0\left( x - x^2 + \frac{1}{6}x^3 - \frac{1}{90}x^4 + \cdots \right),\\[4pt]
z(x) &= a_0\left( -x^2 + \frac{1}{3}x^3 - \frac{1}{30}x^4 + \cdots \right).
\end{align*}

\subsection{Part(d): Compare four-term series with numerical solution}

In order to assess the accuracy of the four-term Frobenius approximation obtained in
part(c), we compare it with a numerical solution of the system
\[
x\,\dfrac{dy}{dx} - y = z, \quad
2x\,\dfrac{dz}{dx} - 3z = -xy,
\]
on a small interval to the right of the singular point $x=0$.\\ \\
Since both equations contain the singular coefficient $\frac{1}{x}$, the numerical integration cannot start exactly at $x=0$. Therefore, we choose a small value $x_0=10^{-3}$ and initialise the numerical solver using the truncated Frobenius series
\begin{align*}
y(x_0) &\approx a_0 x_0 - a_0 x_0^2 + \frac{a_0}{6} x_0^3 - \frac{a_0}{90}x_0^4,\\[4pt]
z(x_0) &\approx -a_0 x_0^2 + \frac{a_0}{3}x_0^3 - \frac{a_0}{30}x_0^4.
\end{align*}
We then solve the system numerically using \texttt{solve\_ivp} on the interval
$[x_0,0.5]$.  The numerical solution and the four-term series approximation for
$y(x)$ and $z(x)$ are plotted in Figure~\ref{fig:comparison}.

\subsubsection{Observation}
As shown in the figure, the four-term Frobenius approximation agrees extremely well with the numerical solution for small $x$.  This is expected, since the Frobenius series provides a local power-series representation about the singular point $x=0$.

\subsubsection{Conclusion}
This shows that the Frobenius method provides a reliable local approximation near the singular point, and the shortened series matches the true solution very well when $x$ is small.
\begin{figure}[h]
    \centering
    \includegraphics[width=0.8\textwidth]{comparison_plot.jpg}
    \caption{Comparison between the numerical solution and the four-term Frobenius
    approximation of part (c).  The two curves coincide very well near $x=0$.}
    \label{fig:comparison}
\end{figure}

\subsection{Result: Question A}
By analysing the lowest–order balance of the system near the singular point $x=0$, we identified the possible relationships between the leading exponents of the Frobenius series for $y$ and $z$. This led to two admissible exponent pairs, $(p,q)=(1,2)$ and $(\tfrac{3}{2},\tfrac{3}{2})$. Examining the structure of the series shows that only the pair $(1,2)$ produces a solution that is analytic at the expansion point, whereas the branch $(\tfrac{3}{2},\tfrac{3}{2})$ contains half–integer powers and is therefore non-analytic at the origin.  \\ \\
Using the analytic branch, the coefficients in the Frobenius expansion can be determined recursively, yielding the four-term approximations
\[
y(x)=a_0x - a_0x^2 + \tfrac{a_0}{6}x^3 - \tfrac{a_0}{90}x^4 + \cdots,
\qquad
z(x)=-a_0x^2 + \tfrac{a_0}{3}x^3 - \tfrac{a_0}{30}x^4 + \cdots,
\]
expressed in terms of the single free constant $a_0$.
The numerical solution by using the series at a small $x_0 > 0$ agrees extremely well with the cut-off expansion for small value of $x$. This demonstrates that the Frobenius method provides an accurate and reliable local representation of the true solution near the singular point.

\section{Question B: Heating a square plate}
The problem is defined by the following Partial Differential Equation and its associated conditions
\begin{align*}
    \text{PDE:} \quad & \frac{\partial u}{\partial t} = \frac{\partial^2 u}{\partial x^2} + \frac{\partial^2 u}{\partial y^2} \\[4pt]
    \text{Domain:} \quad & 0 < x < L,\quad 0 < y < L,\quad t > 0 \\[4pt]
    \text{Boundary conditions:} \quad & 
    \begin{aligned}
        &u(0,y,t) = 0, &\quad &u(L,y,t) = 0 \\[4pt]
        &u(x,0,t) = 0, &\quad &u(x,L,t) = 1
    \end{aligned} \\[4pt]
    \text{Initial condition:} \quad & u(x,y,0) = 0
\end{align*}

\subsection{Part(a): Steady-State Solution $U(x,y)$}

\subsubsection{Problem Definition}
The steady-state solution $U(x,y)$ is the temperature profile as $t \to \infty$, at which point $u_t = 0$. The problem reduces to the Laplace equation
\[
\nabla^2 U = U_{xx} + U_{yy} = 0.
\]
with the original boundary conditions
\begin{align*}
    U(0,y) = 0, \quad U(L,y) = 0, \quad U(x,0) = 0, \quad U(x,L) = 1.
\end{align*}

\subsubsection{Separation of Variables}
We solve this using separation of variables, $U(x,y) = X(x)Y(y)$.
Then we obtain this equation
\[
\frac{X''(x)}{X(x)} = -\frac{Y''(y)}{Y(y)} = -\lambda.
\]
Given the homogeneous Dirichlet boundary conditions (DBC) $X(0)=0$ and $X(L)=0$, the eigenvalues and eigenfunctions are given by
\[
    \lambda_n = \left(\frac{n\pi}{L}\right)^2, \quad X_n(x) = b_n\sin\left(\frac{n\pi x}{L}\right), \quad n=1, 2, 3, \dots.
\]
For the $y$-dependence, the separated equation is
\[
    Y''(y) - \lambda_n Y(y) = 0.
\]
Since $\lambda_n = (n\pi/L)^2 > 0$, the general solution of $Y$ is
\[
    Y_n(y) = A_n \cosh\left( \frac{n\pi y}{L} \right) + B_n \sinh\left( \frac{n\pi y}{L} \right).
\]
Applying the homogeneous boundary condition, $U(x,0) = 0$, which implies $Y(0) = 0$:
\[
    Y_n(0) = A_n(1) + B_n(0) = 0 \implies A_n = 0.
\]
Thus, the eigenfunction for $Y$ is
\[
    Y_n(y) = B_n\sinh\left( \frac{n\pi y}{L} \right).
\]

\subsubsection{General Solution and Coefficients}
The general steady-state solution is constructed as a series sum of product solutions $X_n(x)Y_n(y)$:
\[
    U(x,y) = \sum_{n=1}^{\infty} C_n \sin\left( \frac{n\pi x}{L} \right) \sinh\left( \frac{n\pi y}{L} \right).
\]
Applying $U(x,L) = 1$
\[
    1 = \sum_{n=1}^{\infty} \left[ C_n \sinh(n\pi) \right] \sin\left( \frac{n\pi x}{L} \right).
\]
This is a Fourier sine series expansion of the constant function $f(x) = 1$ over the interval $[0, L]$. Using the orthogonality of sine functions, we calculate
\begin{align*}
    C_n \sinh(n\pi) &= \frac{2}{L} \int_{0}^{L} (1) \cdot \sin\left( \frac{n\pi x}{L} \right) \, dx.
\end{align*}
Solving for $C_n$
\[
    C_n = \frac{2[1 - (-1)^n]}{n\pi \sinh(n\pi)}.
\]
The final form of the steady-state solution is
\[
U(x,y) = \sum_{k=1}^{\infty} \frac{4}{(2k-1)\pi \sinh((2k-1)\pi)} \sin\left(\frac{(2k-1)\pi x}{L}\right) \sinh\left(\frac{(2k-1)\pi y}{L}\right).
\]


\subsection{Part(b): Initial boundary value problem for $v(x,y,t)$}

Let $u(x,y,t) = U(x,y) + v(x,y,t)$. We determine the Initial Boundary Value Problem for the transient function $v(x,y,t)$.

\subsubsection{Partial Differential Equation}
Substituting the decomposition into the heat equation $u_t = \nabla^2 u$
\[
    \frac{\partial}{\partial t}(U + v) = \nabla^2 (U + v).
\]
Since $U(x,y)$ is the steady-state solution, $U_t = 0$ and $\nabla^2 U = 0$. The equation simplifies to:
\[
    v_t = \nabla^2 v = v_{xx} + v_{yy}.
\]

\subsubsection{Boundary Conditions}
The boundary conditions for $v$ are derived by subtracting the steady-state boundary values from the original boundary values ($v = u - U$)
\begin{align*}
    v(0,y,t) &= u(0,y,t) - U(0,y) = 0 - 0 = 0, \\[4pt]
    v(L,y,t) &= u(L,y,t) - U(L,y) = 0 - 0 = 0, \\[4pt]
    v(x,0,t) &= u(x,0,t) - U(x,0) = 0 - 0 = 0, \\[4pt]
    v(x,L,t) &= u(x,L,t) - U(x,L) = 1 - 1 = 0,
\end{align*}
Thus, $v(x,y,t)$ satisfies homogeneous Dirichlet boundary conditions on all sides.

\subsubsection{Initial Condition}
The initial condition at $t=0$ is given by
\[
    v(x,y,0) = u(x,y,0) - U(x,y).
\]
Since $u(x,y,0) = 0$, this becomes
\[
    v(x,y,0) = -U(x,y).
\]

\subsubsection{Summary of IBVP}
The complete problem for $v(x,y,t)$ is
\begin{equation*}
\begin{cases}
    v_t = v_{xx} + v_{yy} \\
    v(0,y,t) = 0, \quad v(L,y,t) = 0 \\
    v(x,0,t) = 0, \quad v(x,L,t) = 0 \\
    v(x,y,0) = -U(x,y)
\end{cases}
where\quad 0 < x < L, \, 0 < y < L, \, t > 0
\end{equation*}

\subsection{Part(c): Boundary value problems for $X(x)$ and $Y(y)$}
Using separation of variables $v(x,y,t) = X(x)Y(y)T(t)$. Substituting this into differential equation\quad$v_t = v_{xx} + v_{yy}$
\[
    X Y T' = X'' Y T + X Y'' T.
\]
Dividing by $X Y T$ to separate the variables
\[
    \frac{T'}{T} = \frac{X''}{X} + \frac{Y''}{Y}.
\]
Notice that $X(x)$ and $Y(y)$ are subject to homogeneous Dirichlet boundary conditions, then we have
\[
    \frac{X''(x)}{X(x)} = -\lambda^2 \implies X''(x) + \lambda^2 X(x) = 0.
\]
With boundary conditions $X(0)=0, X(L)=0$.
\[
    \frac{Y''(y)}{Y(y)} = -\mu^2 \implies Y''(y) + \mu^2 Y(y) = 0.
\]
With boundary conditions $Y(0)=0, Y(L)=0$.
Substituting them back into the equation
\[
    \frac{T'(t)}{T(t)} = -\lambda^2 - \mu^2 = -(\lambda^2 + \mu^2).
\]
Thus, the time equation is:
\[
    T'(t) + (\lambda^2 + \mu^2)T(t) = 0.
\]

\subsection{Part(d): Complete solutions for $v(x,y,t)$ and  $u(x,y,t)$}

\subsubsection{Derivation from Separation of Variables}
From part 3.3, using separation of variables $v(x,y,t) = X(x)Y(y)T(t)$, we obtained the following ordinary differential equations
\[
X''(x) = -\lambda^2 X(x), \quad Y''(y) = -\mu^2 Y(y), \quad T'(t) = -(\lambda^2 + \mu^2)T(t).
\]
With boundary conditions
\[
X(0) = X(L) = 0, \quad Y(0) = Y(L) = 0.
\]
Solving these eigenvalue problems, we obtain the eigenfunctions
\[
X_n(x) = \sin\left(\frac{n\pi x}{L}\right), \quad Y_m(y) = \sin\left(\frac{m\pi y}{L}\right),
\]
with eigenvalues
\[
\lambda_n = \frac{n\pi}{L}, \quad \mu_m = \frac{m\pi}{L}.
\]
The temporal component becomes
\[
T_{nm}(t) = \exp\left(-\frac{\pi^2(n^2 + m^2)}{L^2}t\right).
\]
Therefore, the general form of the transient solution is
\[
v(x,y,t) = \sum_{n=1}^{\infty} \sum_{m=1}^{\infty} A_{nm} \sin\left(\frac{n\pi x}{L}\right) \sin\left(\frac{m\pi y}{L}\right) \exp\left(-\frac{\pi^2(n^2 + m^2)}{L^2}t\right).
\]

\subsubsection{Determination of Fourier Coefficients}

To determine the coefficients $A_{nm}$, we use the initial condition $v(x,y,0) = -U(x,y)$, where $U(x,y)$ is the steady-state solution
\[
U(x,y) = \sum_{k=1}^{\infty} \frac{4}{(2k-1)\pi \sinh((2k-1)\pi)} \sin\left(\frac{(2k-1)\pi x}{L}\right) \sinh\left(\frac{(2k-1)\pi y}{L}\right).
\]
Thus, at $t=0$, we have $v(x,y,0)$ which equals to
\[
-U(x,y)=\sum_{n=1}^{\infty} \sum_{m=1}^{\infty} A_{nm} \sin\left(\frac{n\pi x}{L}\right) \sin\left(\frac{m\pi y}{L}\right) .
\]
Using the orthogonality of sine functions, we multiply both sides by $\sin\left(\frac{n\pi x}{L}\right)\sin\left(\frac{m\pi y}{L}\right)$ and integrate over the domain
\[
A_{nm} = -\frac{4}{L^2} \int_0^L \int_0^L U(x,y) \sin\left(\frac{n\pi x}{L}\right) \sin\left(\frac{m\pi y}{L}\right) dxdy.
\]
Substituting the series expansion for $U(x,y)$ and using orthogonality properties, we find that only terms with odd $n$ contribute, and we obtain
\[
A_{(2k-1)m} = -\frac{8 m (-1)^m}{(2k-1) \pi^2 \left( (2k-1)^2 + m^2 \right)}.
\]

\subsubsection{Complete Solution}

The complete temperature distribution is therefore
\[
u(x,y,t) = U(x,y) + v(x,y,t),
\]
\begin{align*}
u(x,y,t) = & \sum_{k=1}^{\infty} \frac{4}{(2k-1)\pi \sinh((2k-1)\pi)} \sin\left(\frac{(2k-1)\pi x}{L}\right) \sinh\left(\frac{(2k-1)\pi y}{L}\right) \\
& + \sum_{k=1}^{\infty} \sum_{m=1}^{\infty} -\frac{8 m (-1)^m}{(2k-1) \pi^2 \left( (2k-1)^2 + m^2 \right)} \sin\left(\frac{(2k-1)\pi x}{L}\right) \sin\left(\frac{m\pi y}{L}\right) \\
& \times \exp\left( -\frac{\pi^2 \left( (2k-1)^2 + m^2 \right) t}{L^2} \right).
\end{align*}

\subsubsection{Complete Solution and Visualization}


The complete solution for the temperature distribution $u(x,y,t)$ is given by
\begin{equation*}
u(x,y,t) = U(x,y) + v(x,y,t).
\end{equation*}
To compute the solution numerically, we truncate the infinite series in both the steady-state and transient parts. For the steady-state solution $U(x,y)$, we retain the first 30 terms in the series (i.e., $k=1$ to $30$). For the transient solution $v(x,y,t)$, we retain the first 15 terms in the $k$-series and the first 10 terms in the $m$-series. These truncations are chosen such that the solution has converged sufficiently, and further terms do not significantly alter the solution.



\begin{figure}[H]
\centering
\includegraphics[width=0.85\linewidth]{2DHeat.png} 
\caption{Temperature distribution evolution in the square plate at key time points: 
early time ($t=0.01$), intermediate time ($t=0.1$), late time ($t=1.0$), 
and steady state ($t \to \infty$). The plate is heated from the top edge while 
the other three edges are maintained at zero temperature, which shows the temperature distribution in the $(x,y)$-plane at four representative time points, demonstrating the evolution from the initial cold state to the final steady-state distribution.}
\label{fig:temperature_evolution}
\end{figure}

\subsection{Result: Question B}

The heat equation on the square plate admits a natural decomposition
\[
u(x,y,t) = U(x,y) + v(x,y,t),
\]
where $U(x,y)$ is the steady-state solution and $v(x,y,t)$ represents the transient decay.

\begin{enumerate}
    \item \textbf{Steady-State Solution.}  
    The steady-state temperature distribution satisfies Laplace's equation with the given boundary conditions and is expressed as
    \[
    U(x,y) = \sum_{k=1}^{\infty} 
    \frac{4}{(2k-1)\pi\sinh((2k-1)\pi)}
    \sin\!\left(\frac{(2k-1)\pi x}{L}\right)
    \sinh\!\left(\frac{(2k-1)\pi y}{L}\right).
    \]
    This describes the final temperature profile as $t \to \infty$.

    \item \textbf{Transient Behaviour.}  
    The transient component decays exponentially in time,
    \[
    v(x,y,t) = \sum_{k=1}^{\infty}\sum_{m=1}^{\infty} 
    A_{(2k-1)m}
    \sin\!\left(\frac{(2k-1)\pi x}{L}\right)
    \sin\!\left(\frac{m\pi y}{L}\right)
    \exp\!\left(-\frac{\pi^{2}\big((2k-1)^{2}+m^{2}\big)}{L^{2}}t\right),
    \]
    showing that all non-steady modes vanish as $t$ increases.

    \item \textbf{Temperature Evolution.}  
    Numerical reconstructions using $k_{\max}=30$ for $U$ and $(k_{\max},m_{\max})=(15,10)$ for $v$ illustrate the diffusion process:
    \begin{itemize}
        \item at $t=0.01$, heating is confined near the top edge;
        \item at $t=0.1$, heat penetrates into the interior;
        \item at $t=1.0$, the solution is close to steady state;
        \item as $t\to\infty$, $u(x,y,t)\to U(x,y)$.
    \end{itemize}

    \item \textbf{Physical Interpretation.}  
    Heat flows from the heated top boundary toward the cooler edges, producing a smooth harmonic temperature field.  
    The steady state reflects a balance between heat input and dissipation through the remaining boundaries.
\end{enumerate}

This analysis shows that separation of variables yields an accurate and rapidly convergent series representation, capturing both the transient evolution and the long-term behaviour of the temperature distribution.


\section{Conclusion}
{
\setlength{\parindent}{2em} 
\hspace*{1.5em}This project used analytical and numerical methods to study both ODEs system and a heat-equation model. For the ODEs, the Frobenius method identified the valid analytic solution, and its truncated series closely matched numerical results near the singular point. For the PDE, separation of variables produced steady-state and transient Fourier solutions, with numerical visualisation confirming convergence to equilibrium.
\setlength{\parindent}{0em}
}
\end{document}
